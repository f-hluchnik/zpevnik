\hyt{kolobezka}
\song{Koloběžka} \interpret{uhlirsverak}{Uhlíř \& Svěrák}
\sub{z pohádky Koloběžka první}

\vers{1}{
\rec{\chord{D}Napadla mě během \chord{A}dneška\\
\chord{D}kolo, kolo, \chord{A}koloběžka.\\
\chord{D}Ta by stála za pokus,\\
\chord{E}polojízda, \chord{A}poloklus.\\
\vinv
Dospělí to neocení,\\
pro ně koloběžka není.\\
Nemáte však ponětí,\\
co to bude pro děti!}
}

\refrainn{1}{
To nás \chord{D}těší, těší, \chord{A}těší, že nás \chord{D}pěší, pěší \chord{A}pěší \chord{D}horko těžko \chord{E}doho\chord{A}ní.\\
\vinv
Pěší \chord{D}strýček s pěší \chord{A}tetou, když se \chord{D}před řídítky \chord{A}pletou, \chord{E}tak se na ně zazvo\chord{A}ní.
}

\vers{2}{
\rec{Setkáme-li se s překážkou, ubrzdíme to podrážkou!\\
Kdo neumí říditi, dá si octan hlinitý!}\\
Dá si octan hlinitý.\\
\vinv
Setkáme-li se s poruchou, dojdeme k cíli po rukou!\\
Setkáme-li se s bahnem, okamžitě zahnem'!\\
Okamžitě zahnem'.
}

\refrainn{2}{
Kdo má hlavu těžkou, těžkou, ať to zkusí s koloběžkou, štěstí se mu vrátí zpět.\\
Kdo má vítr kolem uší, komu srdce láskou buší, toho baví, baví svět.
}

\cod{
Kdo má \chord{D}vítr kolem \chord{A}uší, komu \chord{D}srdce láskou \chord{A}buší, \chord{D}toho baví, \chord{E}baví \chord{A}svět. \chord{F\kk m}\\
\chord{D}Toho baví, \chord{E}baví \chord{A}svět, \chord{F\kk m}\nc\mm \chord{D}toho baví, \chord{E}baví \chord{A}svět.
}
\newpage
