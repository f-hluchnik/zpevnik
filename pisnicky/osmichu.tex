\hyt{osmichu}
\song{O smíchu} \interpret{spiritualkvintet}{Spirituál kvintet}
\ns
\vers{1}{
Kdo z vás se, \chord{D}lidi, umí smát, ne málo, ne moc, akorát,\\
tak ten se má, \textit{tak ten se má, achich ouvej}, \chord{E}vím, co \chord{A\7}dím.\\
Já můžu \chord{D}vám hned vyprávět, co stojí, co stojí světem tenhle svět,\\
jen ten se má, \textit{jen ten se má, achich ouvej}, \chord{A\7}žít je \chord{D}hit.
}

\refrainn{1}{
Proto s mírou, vážení, každý chrup ať vycení\\
a pozor dá, \textit{a pozor dá, achich ouvej}, vím, co dím.\\
Kdo zná tuhle zásadu zepředu, jakož i zezadu,\\
jen ten se má, \textit{jen ten se má, achich ouvej}, žít je hit.
}

\vers{2}{
Za krále Ferdy z Medníku žil jistý pekař perníků,\\
a ten se smál \textit{a ten se smál, chich ouvej}, vím, co dím.\\
Až náhodou jel kolem král, pekař se zrovna smál a smál\\
a už to má \textit{a už to má, achich ouvej}, žít je hit.
}

\refrainn{2}{
Hned rádci králi hlásili, co děsně přesně zjistili,\\
jak se král bál, \textit{jak se král bál, achich ouvej}, vím, co dím.\\
\uv{Máš, králi, zlou veš v kožichu, perníkáři jsi ke smíchu.}\\
A už to má, \textit{a už to má, achich ouvej}, žít je hit.
}

\vers{3}{
O kousek dál žil zase švec, mrzout, bručoun a jezevec,\\
no žádný klaun, \textit{no žádný klaun, achich ouvej}, vím, co dím.\\
Když mával šídlem, šil dratví, tvářil se jak pop nad rakví\\
a už to má, \textit{a už to má, achich ouvej}, žít je hit.
}

\refrainn{3}{
Zas rádci králi hlásili, co děsně přesně zjistili,\\
jak se král bál, \textit{jak se král bál, achich ouvej}, vím, co dím.\\
\uv{Co máš korunu na čele, netvářil se dost vesele.}\\
A už to má, \textit{a už to má, achich ouvej}, žít je hit.
}

\vers{4}{
Ten smál se moc a ten málo, královi se to nezdálo\\
a už to maj' \textit{a už to maj', achich ouvej}, vím, co dím.\\
Teď škrábají se na hlavě, sedíce spolu v šatlavě,\\
jó, já to znám, \textit{jó, já to znám, achich ouvej}, žít je hit.
} \refsm{1}
\newpage
