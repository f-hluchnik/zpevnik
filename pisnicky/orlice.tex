\hyt{orlice}
\song{Orlice} \interpret{navarova}{Zuzana Navarová}

\vers{1}{
\chord{A}Orlice zas \chord{E}šumí nad spla\chord{F\kk m\7}vem.\\
\chord{D}Odpusť mi, \chord{A}modlím se, ať \chord{E}všechno přesta\chord{C\kk m\7}nem.\\
\chord{C\kk\7}Odpusť mi \chord{D}ke smutku ten \chord{E}chlad, je květen \chord{C\kk m\7}a ne listo\chord{F\kk m\7}pad,\\
odpusť, \chord{A}modlím se, když \chord{E}spáváš na pra\chord{D}vém \chord{D\4}boku, když
}

\vers{2}{
orlice zas šumí nad splavem.\\
Odpusť mi, modlím se, ať všechno přestanem.\\
Odpusť, že nemám na duši smutek, ač králi Artuši\\
prý utek' kůň i s bílým praporem.
}

\refrainn{1}{
\chord{C}Vysoko v \chord{G}horách prší, \chord{A}dolem přelít \chord{D}hvízdák\footnote{Hvízdák je český název některých kachen rodu Anas. Ačkoli se v češtině používá jako název rodový, představuje podrod Mareca.}.\\
Vysoko v horách prší na pěvuší\footnote{Tento výraz nejspíše souvisí s ptákem jménem pěvuška, ačkoli ze slovotvorného hlediska by správné přídavné jméno od slova pěvuška mělo znít pěvuščí (žádný pták \uv{pěvuše} zjevně neexistuje, vyskytuje se jen ženské jméno Pěvuše). Na většině webových stránek publikujících text této písně je na tomto mísně použit neexistující výraz \uv{pyruších}. Takováto nepochopení textu bývají označována jako mondegreeny. Tento termín údajně pochází od americké spisovatelky Sylvie Wright, která ho poprvé užila v roce 1954, když popisovala, jak v textu jedné skotské balady jako dítě opakovaně slyšela Lady Mondegreen místo náležitého laid him on the green. Do širšího povědomí pak různé další mondegreeny rozšířil především novinář Jon Carroll.} hnízda.\\
Vysoko v horách prší, Dáša ruší, hvízdá\footnote{Odkaz na reálnou společenskou událost, kdy Dagmar Havlová v roce 1998 pískala při volbě prezidenta ve Španělském sále na Pražském hradě poté, co republikánský poslanec Jan Vik konstatoval, že Václav Havel není legitimně zvoleným prezidentem České republiky.}.\\
\chord{C}Vysoko v \chord{G}horách horko, v \chord{A}autobuse \chord{D}bonbón v puse, a ty \chord{H}hvízdáš \chord{E}s ní, zdaleka
}

\vers{3}{
orlice zas šumí nad splavem.\\
Odpusť mi, modlím se, ať všechno přestanem.\\
Odpusť mi, ať nemám nikdy strach, kéž dýchám na zobáčcích vah,\\
odpusť, orlice zas šumí nad splavem.
}

\refrainn{2}{
Vysoko v horách prší, voda na tři couly.\\
Vysoko v horách prší, Dáša už se choulí.\\
Vysoko v horách prší, na kalhotách bouli.\\
Vysoko v horách horko, v autobuse sucho v puse, co si počneš s ní, zdaleka
}

\vers{4}{
= \mm\textbf{2.}
}
\ns

\vers{5}{
= \mm\textbf{3.}
}
\newpage
