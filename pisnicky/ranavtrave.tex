\hyt{ranavtrave}
\song{Rána v trávě} \interpret{zalman}{Žalman \& spol.}

\note{capo 3}

\refrain{
\chord{Am}Každý ráno boty \chord{G}zouval, \chord{Am}orosil si nohy \chord{G}v trávě,\\
\chord{Am}že se lidi mají rádi, \chord{G}doufal, \chord{Am}a pro\chord{Em}citli \chord{Am}právě.\\
\chord{Am}Každý ráno dlouze \chord{G}zíval, \chord{Am}utřel čelo do ru\chord{G}kávu,\\
\chord{Am}a při chůzi tělem sem tam \chord{G}kýval, \chord{Am}před se\chord{Em}bou sta \chord{Am}sáhů.\footnote{Sáh je historická antropometrická délková míra odvozená od rozpětí rozpažených rukou dospělého člověka. Staročeský sáh odpovídal 1,793 m. V angloamerické měrné soustavě se sáh nazývá \uv{fathom}, patří k nautickým jednotkám a má délku 1,8288 m.}
}

\vers{1}{
\chord{C}\nc Poznal \chord{G}Mora\chord{F}věnku \chord{C}krásnou,\\
\chord{Am}\nc a ví\chord{G}nečko \chord{C}ze zlata.\\
\chord{C}\nc V Čechách \chord{G}slávu \chord{F}muzi\chord{C}kantů\\
\chord{Am}\nc uma\chord{Em}zanou \chord{Am}od bláta.
} \refsm{}

\vers{2}{
Toužil najít studánečku\\
a do ní se podívat,\\
by mu řekla, proč, holečku,\\
musíš světem chodívat.
}

\vers{3}{
Studánečka promluvila:\\
\uv{To ses musel nachodit!\\
Abych já ti pravdu řekla,\\
měl ses jindy narodit.}
} \refsm{}

\cod{\rep{před sebou sta sáhů}} \note{fade out}
\newpage
