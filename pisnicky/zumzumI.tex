\hyt{zumzumI}
\song{Zum zum I} \interpret{paveldobes}{Pavel Dobeš}

\vers{1}{
\chord{G}Chemie se stala mám \chord{D\7}hobby, nepovím vám už, od které \chord{G}doby.\\
\chord{G}Za vesnicí v polích tratím \chord{D\7}hodiny a pozoruji chemizaci \chord{G}pastviny.\\
\chord{G}Pozoruji jalovice, \chord{D\7}voly, jak obcházejí vyvinuté \chord{G}stvoly.
}

\refrainn{1}{
\chord{G}Zum zum zum \chord{D\7}zum, a nejde mi to do kebule,\\
\chord{G}zum zum zum \chord{D\7}zum, a nejde mi to na ro\chord{G}zum.
}

\vers{2}{
Každý přece ví, že tahle tráva delikátně chutná a je zdravá.\\
Do soutěže přihlásit se nebojím s trávou, která rostla někde na hnoji.\\
Přesto tady ovšem roste ladem a skot se vrací do JZD hladem.
} \refsm{}

\vers{3}{
Vyhledal jsem tudíž agronoma, měl jsem štěstí, chytil jsem ho doma.\\
Proč výsledky práce k cíli nevedou, vyřešili předevčírem s předsedou.\\
Pak vysvětlil mi všechno fundovaně, tak dneska mohu agitovat za ně.
}

\refrainn{2}{
Zum zum zum zum, protože když to nejde do kebule,\\
zum zum zum zum, tak to leze na rozum.
}

\vers{4}{
Je to tím, že dobytek je hloupý, a právě proto mívá svoje roupy.\\
Kdyby četl magazíny, noviny, nedělal by už takové kraviny.\\
S rozesmátou tlamou bral by žrádlo a nikdy by mi ani nenapadlo.
}

\refrainn{3}{
Zum zum zum zum, že mi to nejde do kebule,\\
zum zum zum zum, že mi to nejde na rozum.
}
\newpage
