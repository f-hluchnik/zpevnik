\hyt{bluesfolsomskeveznice}
\song{Blues folsomské věznice} \interpret{tucny}{Greenhorns}
\hyl{folsomprisonblues}{Folsom Prison Blues}

\vers{1}{
Můj \chord{E}děda bejval blázen, texaskej ahasver\footnote{Ahasver je postava \uv{věčného Žida}, který kvůli svému provinění musí až do dne posledního soudu bloudit po zemi. Podle legendy Ahasver udeřil Ježíše Krista při výslechu u Kaifáše. Podle jiné verze se mu vysmíval, když nesl kříž, slovy: \uv{Tak ty tvrdíš, že se vrátíš.} A Ježíš odpověděl: \uv{Ano, a ty tu na mě počkáš.} A od té doby musí Ahasver bloudit po zemi a čekat. Objevuje se v dílech Byrona či Vrchlického, též Apollinaira -- Pražský chodec (Isaac Lakedem), nebo v novele Krysař od Viktora Dyka. V přeneseném významu jméno označuje nespokojeného, věčně se toulajícího člověka, zarostlého a nuzného vzhledu, který nenachází uspokojení a uplatnění.},\\
a na půdě nám po něm zůstal \chord{E\7}ošoupanej kvér.\\
Ten \chord{A}kvér obdivovali všichni kámoši z oko\chord{E}lí\\
a \chord{H\7}máma mi říkala: \uv{Nehrej si s tou pisto\chord{E}lí!}
}

\vers{2}{
Jenže i já byl blázen, tak zralej pro malér,\\
a ze zdi jsem sundával tenhleten dědečkův kvér.\\
Pak s kapsou vyboulenou chtěl jsem bej chlap all right\\
a s holkou vykutálenou hrál jsem si na Bonnie and Clyde\footnote{Bonnie Parker a Clyde Barrow byli známí američtí zločinci, kteří se svým gangem procestovali střední část Spojených států amerických v době světové hospodářské krize v letech 1929–1933.}.
}

\vers{3}{
Ale udělat banku, to není žádnej žert,\\
sotva jsem do ní vlítnul, hned zas vylít' jsem jak čert.\\
Místo jako kočka, já utíkám jak slon,\\
takže za chvíli mě veze policejní anton.
}

\vers{4}{
Teď vokno mřížovaný mně říká, že je šlus,\\
proto tu ve věznici zpívám tohle Folsom blues.\\
Pravdu měla máma, radila: \uv{Nechoď s tou holkou!}\\
a taky mně říkávala: \uv{Nehrej si s tou pistolkou!}
}
\newpage
