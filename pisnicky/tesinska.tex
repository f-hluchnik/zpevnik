\hyt{tesinska}
\song{Těšínská} \interpret{nohavica}{Jaromír Nohavica}

\vers{1}{
\chord{Am}Kdybych se narodil \chord{Dm}před sto lety \chord{F}\nc\chord{E}v tomhle \chord{Am}městě,\chord{Dm}\nc\chord{F}\nc\chord{E}\nc\chord{Am}\\
\chord{Am}u Larischů na zahradě \chord{Dm}trhal bych květy \chord{F}\nc\chord{E}své ne\chord{Am}věstě.\chord{Dm}\nc\chord{F}\nc\chord{E}\nc\chord{Am}\\
\chord{C}Moje nevěsta by \chord{Dm}byla dcera ševcova, \chord{F}z domu Kamiňskich \chord{C}odněkud ze Lvova,\\
\chord{C}kochałbym ja i \chord{Dm}pieścił, \chord{F}chy\chord{E}ba lat \chord{Am}dwieście\footnote{Miloval bych a hladil, chybí dvěstě let.}.\chord{Dm}\nc\chord{F}\nc\chord{E}\nc\chord{Am}
}

\vers{2}{
Bydleli bychom na Sachsenbergu v domě u žida Kohna,\\
nejhezčí ze všech těšínských šperků byla by ona.\\
Mluvila by polsky a trochu česky, pár slov německy, a smála by se hezky,\\
jednou za sto let zázrak se koná, zázrak se koná.
}

\vers{3}{
Kdybych se narodil před sto lety, byl bych vazačem knih.\\
U Prohazků dělal bych od pěti do pěti a sedm zlatrel za to bral bych.\\
Měl bych krásnou ženu a tři děti, zdraví bych měl a bylo by mi kolem třiceti,\\
celý dlouhý život před sebou, celé krásné dvacáté století.
}

\vers{4}{
Kdybych se narodil před sto lety, v jinačí době,\\
u Larischů na zahradě trhal bych květy, má lásko, tobě.\\
Tramvaj by jezdila přes řeku nahoru, slunce by zvedalo hraniční závoru\\
a z oken voněl by sváteční oběd.
}

\vers{5}{
Večer by zněla od Mojzese melodie dávnověká.\\
Bylo by léto tisíc devět set deset, za domem by tekla řeka.\\
Vidím to jako dnes, šťastného sebe, ženu a děti a těšínské nebe.\\
Ještě, že člověk nikdy neví, co ho čeká.
}

