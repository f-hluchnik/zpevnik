\hyt{totenkratvsedesatomosmom}
\song{To tenkrát v šedesátom osmom}

\vers{1}{
\chord{D}Mám v Plzni kámoše, už je to řada let, \chord{G}vždycky rád slyším plzeňskej dialekt\\
a \chord{G}když tam jedu čundrem, \chord{B}dám o \chord{A\7}sobě \chord{D}vědět.\\
Řekne mi, \chord{D}usárnu, tu si tutam dej, na \chord{G}tuten gauč, tam si nesedej, tam už\\
\chord{D}přes dvacet let \chord{B}nesmí \chord{A\7}nikdo \chord{D}sedět.\\
\vinv
Ani já tam nesedám, to se nevyplácí, tam seděli tenkrát v srpnu Rusáci, no, a jak sem vlítli, děly se psí kusy.\\
Tátu vytáh' z úkrytu ten ataman, tlustým zadkem rozsednul nám otoman\\
a od těch dob v něm máme blechy, šváby, rusy.
}

\refrain{1}{
To tenkrát v šedesátom \chord{D}osmom, když Rus nás osvobodil \chord{A\7}po svom\\
s tanky a děly, na čapkách rudý hvězdy \chord{D}měli.\\
Dost splihlý \chord{D}krovky měli soldáti u Ško\chord{A\7}dovky\\
a místní krásky zpívaly sborem Běž domů, Ivane, čeká tě \chord{A\7}Nataša, Nataša, Nata\chord{D}ša.
}

\vers{2}{
Před pár lety byl kámoš u pomníku, rejhy od bot zbyly jen na chodníku,\\
když ho k autu vlekli, ruce měl za záda.\\
Je to dávno, čas tak rychle utíká, jeho dcera měla odznak Fučíka\\
a s ní socialistické práce celá brigáda.\\
Tenkrát v srpnu byla ještě strašně malá, jak kotě průvodce potřebovala,\\
průvode, jenž by ji vedl po vlastním osudu.\\
Tak ji marně lákali na diskohrátky, zbyly po ní jen dva uplakaný řádky,\\
v tomhle blbým státě já už dál žít nebudu.
}

\refrainn{2}{
Já trvám na tom, a na to nemusím být Platon,\\
že dávno měli sedět už doma na prdeli.\\
My jsme ty druhý, my máme k hvězdám rádi pruhy\\
na věčný čásky zpíváme sborem Škoda lásky\dots
}
\newpage
