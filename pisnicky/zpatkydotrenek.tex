\hyt{zpatkydotrenek}
\song{Zpátky do trenek} \interpret{paveldobes}{Pavel Dobeš}
\ns

\refrainn{1}{
Tu du \chord{D}du, tu du \chord{Em}du, tu du \chord{A\7}du, tu du \chord{D}du.
} $\times 2$
\ns

\vers{1}{
Jak z \chord{D}vypitého piva láhev se vrací, jak \chord{Em}od jihu ptáci, jak do potoka raci,\\
\chord{A\7}jako jaro do krajiny modrých \chord{D}pomněnek.\chord{(E\7)}\\
Jak vracejí se investice například z ropy, jak východ se vrací do Evropy,\\
my ze slipů vracíme se zpátky do trenek.
}

\lsp{12}

\vers{2}{
Kdo chce s námi světem jít, rovnou nohou vykročit,\\
po dlouhé noci najít zase den.\\
Mít hlavu plnou nových myšlenek, ten ať se vrátí do trenek,\\
ať se přidá k dlouhé řadě slavných jmen.
}

\lsp{12}

\refrainn{2}{
Byl to \chord{Hm}Shane\footnote{hlavní hrdina westernu Shane}, byl to Shane, \chord{Em}Tom Sawyer i Chamberlain,\footnote{Austen Chamberlain, (1863--1937) britský státník a nositel Nobelovy ceny za mír}\\
\chord{A\7}Hillary,\footnote{Edmund Hillary, (1919--2008) novozélandský horolezec a průzkumník, jako první vystoupil na horu Mount Everest (1953) se šerpou Tenzingem Norgayem} Amundsen,\footnote{Roald Amundsen, (1872--1928) norský polární badatel, jako první dosáhl jižního pólu a proplul severozápadní cestou kolem Ameriky} \chord{D}seržant Pepper\footnote{Kapelník fiktivní kapely Sgt. Pepper's Lonely Hearts Club Band, alter ega Beatles. Tato fiktivní kapela jim měla dát svobodu v~hudebním experimentování a~oprostit je od jejich obrazu jako Beatles. Paul McCartney přišel s myšlenkou na píseň s vojenskou kapelou z Eduardovského období (1901--1910, přišlo po Viktoriánském období), jejíž název byl vybrán ve stylu tehdejších skupin ze San Francisca, například Big Brother and the Holding Company. Samotné jméno seržanta Peppera je podle McCartneyho slovní hříčkou vzniklou ze slov \uv{salt and pepper}.} \chord{A\7}Manfred Mann.\footnote{(1940) zakládající člen skupin Manfred Mann a Manfred Mann's Earth Band}\\
\chord{D}Doktor Jekyll i pan Hyde a \chord{Em}každý, kdo chce být allright,\\
\chord{A\7}na doma i na venek, \chord{D}zpátky \chord{A\7}do tre\chord{D}nek.
} \refsm{1}

\lsp{12}

\vers{3}{
Když s dámou chceš prožít něžný sen, pak přes balkon seš vyhoštěn\\
a ve slipech jdeš noční ulicí.\\
V trenkách je to hned jiná, v trenkách jdeš jak z kasina,\\
už ti chybí jenom chleba s hořčicí.
} \refsm{1}

\lsp{12}

\vers{4}{
Svoboda, svoboda, volnost, rovnost, pohoda,\\
po dlouhé noci přišel nový den.\\
Nenávist a násilí jsme pohřbili a zapili,\\
defilujem v dlouhé řadě slavných jmen.
}

\lsp{12}

\refrainn{3}{
Byl to Shane, byl to Shane, Tom Sawyer i Chamberlain,\\
Lomikar, Kozina, i Vlasta Redl ze Zlína.\\
Nosí je pražská Sparta, Timur\footnote{Tamerlán, též Timur nebo Tímur Lenk byl turkitský vojevůdce a krutý dobyvatel. Z turečtiny \textit{timur} znamená železo a \textit{lenk} chromý či kulhavý.} a celá jeho parta,\\
na doma i na venek, zpátky do trenek.
}
\newpage
