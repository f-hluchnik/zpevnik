\hyt{mobilizace}
\song{Mobilizace} \interpret{dobes}{Pavel Dobeš}

\vers{1}{
Dovo\chord{A}lej' mi, \chord{D}pane gene\chord{A}rále, abych \chord{A}je uc\chord{E}tivě pozdra\chord{A}vil \chord{E}\nc\chord{A}\\
a z e\chord{A}ventuelní \chord{D}války nena\chord{A}dálé, abych \chord{A}se jim \chord{E}předem omlu\chord{A}vil. \chord{E}\nc\chord{A}\\
Kdyby \chord{D}zítra namísto sní\chord{A}daně zača\chord{E}lo se mobilizo\chord{A}vat,\chord{E}\\
těžko \chord{A}bych moh' \chord{D}kráčet odhod\chord{A}laně do vál\chord{A}ky se \chord{E}zaangažo\chord{A}vat. \chord{E}\nc\chord{A}
}

\vers{2}{
Museli by nejdřív přemluvit mou ženu, kdyby se jí od snídaně zved'.\\
Nerada mě pouští do terénu, to bych se bál vrátit z války domů zpět.\\
Co by tomu řekli moji kluci, kdybysem se válce věnoval?\\
Vždycky jsem měl tisíc výmluv, hergot, kruci, než abych si s nimi na vojáky hrál.
}

\vers{3}{
Co by tomu řeklo moje zdraví? To když se zkazí, už se nespraví.\\
Vždyť i na klinikách lékaři, když slaví, tak připíjejí hlavně na zdraví.\\
Co by řek' můj nadřízený z práce, ten, když mě pět minut nevidí,\\
už to pozoruji ve obálce ztrátou výkonnostních prémií.
}

\vers{4}{
Co by řek' můj přítel z Marylandu, kdybych tam někde v Americe stanoval?\\
Často píše sice, přijeď o víkendu, ale s jakou by mě asi přivítal?\\
Těžko bych moh' kráčet odhodlaně a do války ruksak pakovat,\\
kdyby zítra namísto snídaně začalo se mobilizovat.
}

\vers{5}{
Závěrem chci poníženě prosit, kdyby si chtěli najít jinou zábavu.\\
U nás nemá kdo v kýblech maltu nosit v akci Z\footnote{Akce Z byla v dobách komunistického režimu v Československu neplacená pracovní činnost obyvatel. Oficiálně se jednalo o dobrovolnou, bezplatnou práci odváděnou zpravidla mimo pracovní dobu většinou o sobotách. Přestože se mělo jednat o dobrovolnou činnost, účast na akcích byla dokumentována a s občany, kteří se odmítli zúčastnit, byly vedeny pohovory.} za novou Ostravu.\\
Venku by jim zčervenaly tváře, lid by na ně přestal hledět úkosem,\\
já bych nyní nečuměl do kalamáře a žili bychom všichni rovnou za nosem.
}
\newpage
