\hyt{tresnicky}
\song{Třešničky} \interpret{spiritualkvintet}{Spirituál kvintet}
\ns
\vers{1}{
Když \chord{G}Josef veselku \chord{Am\7}chystal,\chord{D}\mm byl už \chord{G}starý, skoro \chord{Am\7}kmet,\chord{D}\\
a \chord{G}dívka Marie \chord{Am\7}čistá,\chord{D}\mm \chord{G}gali\chord{Am\7}lejský \chord{D}květ.
}

\vers{2}{
Jednou se šli spolu projít, Josef nechtěl, ale šel,\\
vždyť byli jen krátce svoji a vzduch plný včel.
}

\vers{3}{
Jednou se šli spolu projít, byl červen, třešní čas,\\
a jak tam pod stromem stojí, zní Mariin hlas:
}

\vers{4}{
\uv{Moh' bys, Josífku, prosím, pár třešní natrhat,\\
já dítě pod srdcem nosím, a co když má hlad?}
}

\refrain{
Tu du \chord{D\7}dum\dots
}

\vers{5}{
Tu Josef si začal rvát vlasy: \uv{Ó, já jsem naletěl,\\
kdopak moh' setřít asi tvojí nevinnosti pel?}
}

\vers{6}{
\uv{Víš, já mám trpělivost svatou, ale, to ti povídám,\\
ať ten, kdo je dítěti tátou, mu podá třešně sám!}
}

\vers{7}{
A zatímco Josef tu běsní, náhle dětský hlásek slyš:\\
\uv{Sehni se, milá třešni, k mojí mamince blíž!}
}

\vers{8}{
A hle, strom se k zemi sklání, Marie trhá s úsměvem\\
a Josef nemá ani zdání, jak z té šlamastyky ven.
}\refsm{}

\vers{9}{
\uv{Můj Bože, já se tolik stydím, ale řekni, Pane náš,\\
kdy chceš přijít mezi lidi, kdypak narodit se máš?}
}

\vers{10}{\uv{Já myslím tak šestého ledna, v době vánoc, v noční čas,\\
až se zjeví hvězda neposedná, tehdy přijdu mezi vás.}
}

\vers{11}{Roky skáčou jako hříbě, ale když se snese sníh,\\
lidé zpívají si příběh o těch třech a o třešních.
}
\newpage
