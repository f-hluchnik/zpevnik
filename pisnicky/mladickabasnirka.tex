\hyt{mladickabasnirka}
\song{Mladičká básnířka}
\interpret{nohavica}{Nohavica}

\ns
\vers{1}{
\chord{G}Mladičká básnířka \chord{Hm}s korálky nad kotní\chord{Em}ky,\mm\chord{D}\\
\chord{G}bouchala na dvířka \chord{Hm}paláce poeti\chord{Em}ky.\mm\chord{D}S někým se\\
\chord{G}vyspala, někomu \chord{D}nedala, láska jako \chord{Em}hobby.\\
\chord{Cm}Pak o tom napsala \chord{D}blues na čtyři \chord{G}doby.\chord{Hm}\nc\chord{Em}\nc\chord{D}
}

\vers{2}{
Své srdce skloňovala podle vzoru Ferlinghetti\footnote{Lawrence Ferlinghetti, \textasteriskcentered 1919, \textdagger 2021, americký básník, dramatik, nakladatel a malíř, vydal například Kvílení Allena Ginsberga, jeho dílo bývá řazeno do beat generation},\\
ve vzduchu nechávala viset vždy jen půlku věty, plná\\
tragiky, plná mystiky, plná splínu,\\
pak jí to otiskli v jenom magazínu.
}

\vers{3}{
Bývala viděna v malém baru u Rozhlasu,\\
od sebe kolena a cizí ruka kolem pasu, trochu se\\
napila, trochu se opila, na účet redaktora,\\
za týden nato byla hvězdou Mikrofóra\footnote{Pořad Československého rozhlasu původně určený pro mladé posluchače.}.
}

\vers{4}{
Pod paží nosila rozepsané rukopisy,\\
ráno se budila vedle záchodové mísy,\\
múzou políbená, životem potřísněná, plná zázraků,\\
a pak ji vyhodili z gymplu a hned nato i z baráku. 
}

\vers{5}{
Ve třetím měsíci dostala chuť na jahody,\\
ale básníci-tatíci nepomýšlej' na rozvody. Cítila\\
u srdce jak po ní přešla železná bota,\\
pak o tom napsala sonet, a ten byl ze života.
}

\vers{6}{
A jednou v pondělí přišla do klubu na koleje,\\
a hlasem nesmělým prosila o text Darmoděje,\\
a jak tak psala, náhle se dala tichounce do pláče,\\
a inkoustové slzy kapaly na její mrkváče.
}
\ns

\cod{
Jó, mladé básnířky,\\
vy mladé básnířky,\\
ach, mladé básnířky, mladé básnířky, básnířky.
}
\newpage
