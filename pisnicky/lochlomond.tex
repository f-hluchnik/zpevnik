\hyt{lochlomond}
\song{Loch Lomond}
\interpret{waldemarmatuska}{Waldemar Matuška}

\vers{1}{
Ten \chord{D}kraj už je \chord{Hm}blízko, a \chord{Em}já cestu \chord{A}znám,\\
je \chord{D}bílá a \chord{Hm}jde skotskou \chord{G}plá\chord{D}ní.\\
Jde \chord{G}úbočím \chord{D}skal a \chord{Em}lučinami \chord{A}tam,\\
kde se \chord{D}nad Loch \chord{D7}Lomond\footnote{Tato píseň je českou verzí skotské lidové písně Bonnie Banks o’ Loch Lomond. Loch Lomond je největší jezero ve Skotsku. Nese jméno po hoře Ben Lomond (gaelsky \textit{Beinn Laomainn} – \uv{hora majáků}). Oblast se původně nazývala \textit{Leamhnachd} (angl. Lennox) a samotné jezero dříve \textit{Loch Leamhna} podle řeky Leven (\uv{voda jilmů}).
} \chord{G}stín měkce \chord{A}sklá\chord{D}ní.
}

\vers{2}{
Ten kout a ty louky a skály mám rád,\\
tu stráň, co se v zálivu koupá.\\
Má láska, má láska se na mě bude smát\\
tam, kde nad Loch Lomond závoj mlh stoupá.
}

\vers{3}{
Ten kraj už je blízko a já cestu znám,\\
je bílá a jde skotskou plání.\\
Jde úbočím skal a lučinami tam,\\
kde se nad Loch Lomond stín měkce sklání.
}

\vers{4}{
Až přijde ten čas, vítr z hor začne vát\\
a v té chvíli kvést bude tráva.\\
Má láska, má láska z ní lůžko bude stlát\\
tam, kde sám Loch Lomond v rákosí spává.
}

\newpage