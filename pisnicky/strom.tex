\hyt{strom}
\song{Strom}

\vers{1}{
\chord{Am}Polní cestou kráčeli \chord{G}šumaři do vísky hrát,\\
\chord{Am}svatby, pohřby, tahle cesta \chord{G}poznala mnohokrát,\\
po \chord{F}jedné svatbě se \chord{G}chudým lidem \chord{Am}synek narodil\\
a \chord{F}táta mu u \chord{G}prašný cesty \chord{E}života strom zasadil.
}

\refrainn{1}{
A on tam \chord{A}stál a koukal \chord{F\kk m}do polí,\\
byl jak \chord{D}král, sám v celém \chord{E}okolí.\\
Korunu \chord{F\kk m}měl, korunu měl, i když ne \chord{D}ze zlata,\\
a jeho \chord{A}pokladem byla \chord{E}tráva střapa\chord{A}tá.
}

\vers{2}{
Léta běží a na ten příběh si už nikdo nevzpomněl,\\
jen košatý strom se u cesty ve větru tiše chvěl,\\
a z vísky bylo město, a to město začlo chtít\\
asfaltový koberec až na náměstí mít.
} \refsm{1}

\vers{3}{
Že strom byl v cestě plánované, to malý problém byl,\\
ostrou pilou se ten problém snadno vyřešil,\\
tak naposled se do nebe náš strom pak podíval\\
a tupou ránu do větvoví snad už ani nevnímal.
}

\refrainn{2}{
Stál tam sám, když koukal do polí\dots
}

\vers{4}{
Při stavbě se objevilo, že silnice bude dál,\\
a tak kousek od nové cesty smutný pařez stál,\\
dětem a výletníkům z výšky nikdo nemával,\\
a přítel vítr si o něm píseň na strništích z nouze hrál.
}

\refrainn{3}{
Jak tam stál a koukal do polí\dots
}
\newpage