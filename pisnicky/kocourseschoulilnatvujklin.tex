\hyt{kocourseschoulilnatvujklin}
\song{Kocour se schoulil na tvůj klín} \interpret{hapkahoracek}{Hapka \& Horáček}

\vers{1}{
Kocour se schoulil na tvůj \chord{C}klín u nohou \chord{G}spí ti dalma\chord{Am}tin\mm \chord{G}\\
\chord{C} v pozici lvů, co zdobí \chord{F}banku i sama noc má na ka\chord{C}hánku,\\
jen ty se ještě bráníš \chord{Dm}spánku a síle \chord{Am}starých \chord{G}vín.
}

\vers{2}{
Jak dlouho to však vydržíš? Už na kostele zlátne kříž.\\
Ne, ani jeho svatá záře nenajde důlek do polštáře,\\
natož pak obrys mužské tváře a ty to dobře víš.
}

\refrainn{1}{
Má \chord{F}krásko s přetěžkými \chord{C}víčky, \chord{F}snad je tvůj rytíř na ces\chord{C}tách,\\
\chord{F}snad si tvou krajkou \chord{C}od spodnič\chord{Am}ky stírá \chord{Dm}prach \chord{G}\nc na ces\chord{C}tách.\chord{F}\nc\chord{C}
}

\vers{3}{
Až přes řeku a do strání se rozléhá tvé volání.\\
I Jonáš ve velrybím břiše by tvoje slova musel slyšet\\
vždyť voláš tišeji než tiše, kdo se ti ubrání?
}

\vers{4}{
Vidíš však, nebo nevidíš? Už na kostele zlátne kříž,\\
ne, ani jeho svatá záře nenajde důlek do polštáře,\\
natož pak obrys mužské tváře a den je blíž a blíž.
}

\refrainn{2}{
Má krásko s přetěžkými víčky, kývou se, kývou, misky vah,\\
na jedné krajka od spodničky, na druhé, na druhé víří prach.
}

\vers{5}{
I ve zvířeném prachu cest vidím tě lampu k oknu nést.\\
Tvá silueta ale bledne, kdo pohlédne, už nedohlédne,\\
snad ještě můra k lampě sedne, já však znám tu lest.
}

\refrainn{3}{
Má krásko s přetěžkými víčky, i když má víčka víří prach,\\
tak po tvé krajce od spodničky teď bych už nerad sáh'.
}
\newpage
