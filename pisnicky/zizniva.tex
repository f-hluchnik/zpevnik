\hyt{zizniva}
\song{Žíznivá} \interpret{horak}{Michal Horák}
\vers{1}{
\chord{E}Od prvního \chord{G\kk}pohledu \chord{A}zdála jste se \chord{Am}milá,\\
\chord{E}my jsme ten stůl \chord{G\kk}vepředu, jenž \chord{A}jste si oblí\chord{Am}bila.\\
\chord{E}Možná vypa\chord{G\kk}dáme jako \chord{A}cháska nezle\chord{Am}tilá,\\
ale \chord{E}velkou žízeň \chord{G\kk}máme, \chord{A}a proto Vám \chord{H}pějem toto:
}

\refrain{
\chord{E}Prosíme, \chord{G\kk}mějte s námi \chord{C\kk m}soucit, naše \chord{E\7}srdce budou \chord{A}tlouci jen, \chord{H}když se napi\chord{E}jem.\\
\chord{E}Vypadáme \chord{G\kk}jako vlezlí \chord{C\kk m}brouci, avšak \chord{E}mějte soucit \chord{A}slečno, jen \chord{H}kapku přeži\chord{E}jem.
}

\vers{2}{
Vážně Vám to za tím barem nehorázně sluší,\\
při té Vaší práci s hadrem srdce nám vždy buší.\\
Říkáte si asi, proč ty smradi dál mě ruší?\\
Avšak naše stovka s Vaší kasou po setkání dávno pasou.
}\refsm{}

\vers{3}{
Na co vy Vám, slečno, od teď byla šála?\\
Naše vroucí vděčnost v zimě by Vás hřála!\\
Vidíme to na Vás, že jste jedna z mála,\\
která chápe, jak to píše v břiše z půllitrové číše.
}\refsm{}
\vnv

\emph{rec:} Tak teda Kofolu.
\newpage
