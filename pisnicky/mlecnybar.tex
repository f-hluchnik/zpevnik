\hyt{mlecnybar}
\song{Mléčný bar} \interpret{uhlirsverak}{Uhlíř \& Svěrák}

\vers{1}{
\chord{A}Když jde naše třída do baru \rec{mléčného,}\\
\chord{A}bere s sebou třídního Já\chord{E}ru \rec{Ječného.}\\
\chord{E}Náš třídní je taky na sladký \rec{pečivo,}\\
\chord{E}má daleko radši oplat\chord{A}ky \rec{než pivo.}
}

\refrain{
\chord{A\7}Dáme si \chord{D}pětatřicet chlebíčků (pětatřicet chlebíčků)\\
je nám jako v nebíč\chord{A}ku. (je nám jako v nebíčku)\\
\chord{A}Pětatřicet kremrolí (pětatřicet kremrolí)\\
co se v puse rozdro\chord{D}lí. (co se v puse rozdrolí)\\
\chord{D}Pak se nese do baru (pak se nese do baru)\\
pětatřicet pohá\chord{A}rů. (pětatřicet pohárů)\\
\chord{A}Když to všichni spapají, (když to všichni spapají)\\
pak ještě \chord{A}pětatřicet kakají, kakají, kakají, \chord{A\7}Ježišmarjá, kaka\chord{D}jí.
}

\vers{2}{
Učitel zná řeči a má z nich \rec{doktorát,}\\
bez řečí však sní pět mandlových \rec{čokolád.}\\
Co je dobrý, to je nezdravý \rec{přátelé,}\\
učitel nám o tom vypráví \rec{vesele.}
} \refsm{}

\vers{3}{
Pro případ, že člověk potká psy \rec{toulavé,}\\
koupíme pár dobrot do kapsy, \rec{do pravé.}\\
Vracíme se zpátky do školy \rec{pěšinou,}\\
celou třídu břicho rozbolí \rec{většinou.}
} \refsm{} Měli jsme\dots
\newpage
