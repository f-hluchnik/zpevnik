\hyt{reknikdetykytkyjsou}
\song{Řekni, kde ty kytky jsou}

\vers{1}{
\chord{G}Řekni, kde ty \chord{Em}kytky jsou, \chord{C}co se s nima \chord{D\7}mohlo stát?\\
\chord{G}Řekni, kde ty \chord{Em}kytky jsou, \chord{C}kde mohou \chord{D\7}být?\\
\chord{G}Dívky je tu \chord{Em}během dne \chord{C}otrhaly \chord{D\7}do jedné,\\
\chord{C}kdo to kdy \chord{G}pochopí, \chord{C}kdo to kdy \chord{D\7}pocho\chord{G}pí?
}

\vers{2}{
Řekni, kde ty dívky jsou, co se s nima mohlo stát?\\
Řekni, kde ty dívky jsou, kde mohou být?\\
Muži si je vyhlédli, s sebou domů odvedli,\\
kdo to kdy pochopí, kdo to kdy pochopí?
}

\vers{3}{
Řekni, kdy ti muži jsou, co se k čertu mohlo stát?\\
Řekni, kde ti muži jsou, kde mohou být?\\
Muži v plné polní jdou, do války je zase zvou,\\
kdo to kdy pochopí, kdo to kdy pochopí?
}

\vers{4}{
A kde jsou ti vojáci, lidi, co se mohlo stát?\\
A kde jsou ti vojáci, kde mohou být?\\
Řady hrobů v zákrytu, meluzína kvílí tu,\\
kdo to kdy pochopí, kdo to kdy pochopí?
}

\vers{5}{
Řekni, kde ty hroby jsou, co se s nimi mohlo stát,\\
řekni, kde ty hroby jsou, kde mohou být?\\
Co tu kytek rozkvétá, od jara až do léta,\\
kdo to kdy pochopí, kdo to kdy pochopí?
}

\vers{6}{
=\mm \textbf{1.}
}

\newpage
