\hyt{cernadira}
\song{Černá díra} \interpret{plihal}{Karel Plíhal}

\vers{1}{
\chord{G}Mívali jsme \chord{D}dědečka, \chord{C}starého už \chord{G}pána,\\
\chord{G}stalo se to v \chord{D}červenci, \chord{C}jednou časně \chord{D}zrá\chord{G}na.\\
\chord{Em}Šel do sklepa \chord{C}pro vidle, \chord{A}aby seno \chord{D}sklízel,\\
\chord{G}už se ale \chord{D}nevrátil, \chord{C}prostě někam \chord{D}zmi\chord{G}zel.
}

\vers{2}{
Máme doma ve sklepě malou černou díru,\\
na co přijde, sežere, v ničem nezná míru.\\
Nechoď, babi, pro uhlí, sežere i tebe,\\
už tě nikdy nenajdou příslušníci VB.
}

\vers{3}{
Přišli vědci z daleka, přišli vědci z blízka,\\
babička je nervózní a nás, děti, tříská.\\
Sama musí poklízet, běhat kolem plotny,\\
a děda je ve sklepě nekonečně hmotný.
}

\vers{4}{
Hele, babi, nezoufej, moje žena vaří\\
a jídlo se jí většinou nikdy nepodaří.\\
Půjdu díru nakrmit zbytky od oběda,\\
díra všechno vyvrhne, i našeho děda.
}

\vers{5}{
Tak jsem díru nakrmil zbytky od oběda,\\
díra všechno vyvrhla, i našeho děda.\\
Potom jsem ji rozkrájel motorovou pilou,\\
opět člověk zvítězil nad neznámou silou.
}

\cod{
\chord{A}Dědeček se \chord{E}raduje, \chord{D}že je zase v \chord{A}penzi,\\
\chord{A}teď je naše \chord{E}písnička \chord{D}zralá pro re\chord{E}cen\chord{A}zi.
}

\newpage