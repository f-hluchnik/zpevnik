\hyt{piskajicicikan}
\song{Pískající cikán} \interpret{spiritualkvintet}{Spirituál kvintet}

\vers{1}{
\chord{D}Dívka \chord{Em}loudá se \chord{F\kk m}vinicí, \chord{Em}\nc\chord{D}tam, kde \chord{Em}zídka je \chord{F\kk m} níz -\sm\chord{Em}ká,\\
\chord{D}tam, kde \chord{Em}stráň končí \chord{F\kk m} voní \chord{G}cí, si \chord{D}písnič\chord{G}ku někdo p\chord{D} ís\chord{G}ká. \chord{A}
}

\vers{2}{
Ohledne se a, propána, v stínu, kde stojí líska,\\
švarného vidí cikána, jak leží, písničku píská.
}

\vers{3}{
Chvíli tam stojí potichu, písnička si ji získá,\\
domů jdou spolu ve smíchu, je slyšet cikán, jak píská.
}

\vers{4}{
Jenže tatík, jak vidí cikána, pěstí do stolu tříská,\\
\uv{Ať táhne pryč, vesta odraná, groš nemá, něco ti spíská!}
}

\vers{5}{
Teď smutnou dceru má u vrátek, jen Bůh ví, jak se jí stýská,\\
\uv{Kéž vrátí se mi zas nazpátek ten, který v dálce si píská.}
}

\vers{6}{
Pár šídel honí se po louce, v trávě rosa se blýská,\\
cikán, rozmarýn v klobouce, jde dál a písničku píská.
}

\vers{7}{
Na závěr zbývá už jenom říct, v čem je ten kousek štístka:\\
peníze často nejsou nic, má víc, kdo po svém si píská.
}
\newpage
