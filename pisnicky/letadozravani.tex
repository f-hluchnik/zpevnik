\hyt{letadozravani}
\song{Léta dozrávání}
\vers{1}{
Život je \chord{D\7}pro mne obnošená vesta, vše, co se \chord{G}dalo, dávno už jsem prožil,\\
teď už mi \chord{D\add{7}}zbývá jenom jedna cesta: \chord{D\7}vstříknu si \chord{G}trochu \chord{C}jedu do \chord{E\7}žil.\\
Pár kapek \chord{Am}utrejchu jsem požil a \chord{Cm}zítra zas si pro změnu,\\
v pří\chord{G}padě, že bych ještě ožil, \chord{A\7}kulí hlavu \chord{D\7}proženu.\\
To \chord{G\dm}jsem si \chord{D\7}říkal, když mi bylo dvacet, když nevě\chord{G}děl jsem kudy dál,\\
až dobrý \chord{D\7}přítel vrazil mi pár facek, a pak se \chord{G}to\sm\chord{G\dm}\sm mu\mm \chord{C}se \chord{D\7}mnou \chord{G}smál.
}

\vers{2}{
Dva roky na to -- to jsem se zas věšel pro tu, co měla oči jako mandle,\\
co měla pro mě, když jsem se s ní sešel, polibky s chutí cukrkandle.\\
Pro ni jsem došel k rozhodnutí, že život pes je, buď jak buď,\\
a nikdo, že mě nedonutí, abych ho žil, když nemám chuť.\\
Tu opět přítel objevil se náhle, nadávky jeho vyřklé z plných plic\\
to byly kapky do mé duše zprahlé, a z té mé smrti nebylo zas nic.
}

\vers{3}{
Až jednou: lehl jsem si na koleje, neb jsem měl v duši zas nějaký zmatek,\\
ležel jsem dlouho, vlak však stále nejel a pražce tlačily mě do lopatek.\\
Dobytčí vlak do městských jatek měl ukončit mé trápení,\\
byl všední den a přece svátek, dobytčí vlak měl zpoždění.\\
Čekal jsem zdali přítel se mi zjeví, aby mě vyrval smrti ze klína,\\
nejde a nejde, asi o tom neví, tak jsem se zved a šel jsem do kina.
}

\vers{4}{
\uv{Život je pro mne obnošená vesta, na jeho nudu právem stěžuju si,}\\
to říká sotva jeden člověk ze sta a ostatní se divit musí.\\
A proto kdo si neví rady,ten ať se smíří s osudem,\\
my dokaváde budem tady, nikdy říkat nebudem:\\
\uv{Život je pro mě obnošená vesta, šedá a nudná, jak to račte znát.}\\
to říká sotva jeden člověk ze sta a ostatní se musí smát.
}
\newpage
