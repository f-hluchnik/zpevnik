\hyt{vjednymdumku}
\song{V jednym dumku}

\vers{1}{
\chord{C}V jednym dumku \chord{G}na zarubku,\\
\chord{G\7}měł raz chlopek \chord{C}švarnu robku,\\
\chord{C}a ta robka \chord{G}teho chłopka \chord{G\7}rada nimja\chord{C}ła.\\
\vinv
Bo ten jeji chłopek dobrotisko był,\\
on tej svojej robce všicko porobił,\\
čepani ji pomył, bravkum davał žrať,\\
děcka mušeł kolibať.
}

\vers{2}{
Robił všecko, chovał děcko,\\
taky to był dobrotisko,\\
ale robka teho chłopka rada nimjała.\\
\vinv
Štvero novych šatu, štvero střevice,\\
do kosteła nešła enem k muzice,\\
same šminkovani, sama parada,\\
chłopka nimjała rada.
}

\vers{3}{
A chłopisku, dobrotisku,\\
słze kanum po fusisku,\\
jak to vidi, jak to słyši, jaka robka je.\\
\vinv
Tož dožrało to chłopka, že tak hłupy był,\\
do hospody zašeł vypłatu přepił,\\
a jak přišeł do dom, řval, jak hrom by bil,\\
a tu svoju robku zbił.
}

\vers{4}{
A včił ma robka rada chłopka,\\
jak on piska, ona hopka,\\
Hanysku sem, Hanysku tam, ja tě rada mam.\\
\vinv
Věřte mi, luďkovie, že to tak ma byť,\\
raz za čas třa robce kožuch vyprašiť,\\
a pak je tak hodna jako ovečka\\
a ma rada chłopečka!
}
\newpage
