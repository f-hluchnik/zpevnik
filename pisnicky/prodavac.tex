\hyt{prodavac}
\song{Prodavač} \interpret{tucny}{Michal Tučný}
\ns

\intro{
\chord{G}Pojďte všichni dovnitř, pozvěte si všechny známé,\\
my vám dobrou radu dáme, neboť právě otvíráme,\\
prodáváme, vyděláme, co kdo chcete, tak to máme,\\
co nemáme, objednáme, všechno máme, všechno víme, poradíme, posloužíme!
}
\ns

\vers{1}{
Stál \chord{G}krámek v naší \chord{G\7}ulici, v něm \chord{C}párky, buřty s hořčicí a \chord{D}bonbóny a sýr a sladký \chord{G}mák.\\
Tam \chord{G}chodíval jsem \chord{G\7}potají, tak \chord{C}jak to kluci dělají a \chord{D}ochutnával od okurek \chord{G}lák.\\
A \chord{C}pro mou duši nevinnou pan \chord{G}vedoucí byl hrdinou, když \chord{A\7}po obědě začal prodá\chord{D}vat.\\
Měl \chord{G}jazyk mrštný jako bič a \chord{C}já byl z něho celý pryč a \chord{D}toužil jsem se prodavačem \chord{G}stát.
}

\refrain{
\chord{G}Pět deka, deset deka, dvacet deka, třicet deka,\\
\chord{C}kilo chleba, kilo cukru, jeden rohlík, jedna veka,\\
\chord{D}všechno máme, co kdo chcete, obchod kvete, jen si račte \chord{G}říct.\\
\chord{G}Čtyři kila, deset kilo, dvacet kilo, třicet kilo,\\
\chord{C}navážíme, zabalíme, klaníme se, to by bylo,\\
\chord{D}prosím pěkně, mohu nechat o jedenáct deka \chord{G}víc?
}

\vers{2}{
Já nezapomněl na svůj cíl a záhy jsem se vyučil a moh' být ze mě prodavačů král.\\
Jenomže, jak běžel čas, já zaslechl jsem hudby hlas a znenadání na jevišti stál.\\
I když nejsem králem zpěváků, teď zpívám s partou Fešáků\\
a nikdo vlastně neví, co jsem zač.\\
Mě potlesk hřeje do uší a mnohý divák netuší, že mu vlastně zpívá prodavač.
} \refsm{}

\vers{3}{
Vím že se život rozletí a sním o konci století, kdy nikdo neví, co je chvat a shon.\\
A dětem líčí babička, jak vypadala elpíčka a co byl vlastně starý gramofon.\\
I kdyby v roce dva tisíce byla veta po muzice, obchod je věc stále kvetoucí.\\
Už se vidím, je to krása, ve výloze nápis hlásá:\\
\rec{Michal Tučný, odpovědný vedoucí.}
} \refsm{}
\newpage
