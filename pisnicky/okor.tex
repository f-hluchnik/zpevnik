\hyt{okor}
\song{Okoř}
\vers{1}{
\chord{D}Na Okoř je cesta jako žádná ze sta, \chord{A\7}vroubená je stroma\chord{D}ma.\\
Když jdu po ní v létě, samoten na světě, \chord{A\7}sotva pletu noha\chord{D}ma.\\
\chord{G}Na konci té cesty \chord{D}trnité, \chord{E}stojí krčma jako \chord{A\7}hrad.\\
\chord{D}Tam zapadli trampi, hladoví a sešlí, \chord{A\7}začli sobě noto\chord{D}vat.
}

\refrain{
Na hradě Okoři \chord{A\7}světla už nehoří, \chord{D}bílá paní \chord{A\7}šla už dávno \chord{D}spát.\\
Ta měla ve zvyku, \chord{A\7}podle svého budíku, \chord{D}o půlnoci \chord{A\7}chodit straší\chord{D}vat.\\
\chord{G}Od těch dob, co jsou tam \chord{D}trampové, \chord{E}nesmí z hradu \chord{A\7}pryč.\\ \chord{D}A tak dole v podhradí \chord{A\7}se šerifem dovádí, \chord{D}on jí sebral \chord{A\7}od komnaty klíč.
}

\vers{2}{
Jednoho dne zrána roznesla se zpráva,
že byl Okoř vykraden.\\
Nikdo neví dodnes, kdo to tenkrát odnes, nikdo nebyl dopaden.\\
Šerif hrál celou noc mariáš s bílou paní v kostnici,\\
místo aby hlídal, zuřivě ji líbal, dostal z toho zimnici.
}\refsm{}
\newpage
