\hyt{lojzaaliza}
\song{Lojza a Líza}
\nv\uv{Lojzo? Hej, Lojzo!}\uv{Ano, Lízinko.}\\
\uv{Dojdeš pro vodu?}\uv{Už běžím.}\\
\uv{No proto!}

\vers{1}{
Vědro \chord{E}má ve dně \chord{A}díru, milá Lízo, milá Lízo,\\
vědro \chord{E}má ve dně \chord{A}díru, milá Lízo, \chord{H\7}jak \chord{E}hrom.
}

\begin{multicols}{2}
\vers{2}{
	Tak ji ucpi\dots\\
	\dots ucpi ji.
}

\vers{3}{
	A čím ji mám ucpat\dots\\
	\dots řekni čím.
}

\vers{4}{
	Kouskem slámy\dots\\
	\dots slámy.
}

\vers{5}{
	Jenže sláma je dlouhá\dots\\
	\dots dlouhá.
}

\vers{6}{
	Tak ji utni\dots\\
	\dots utni ji.
}

\vers{7}{
	A čím ji mám utnout\dots\\
	\dots řekni čím.
}

\vers{8}{
	Sekerou\dots\\
	\dots sekerou.
}

\vers{9}{
	Jenže sekera je moc tupá\dots\\
	\dots tupá.
}

\vers{10}{
	Tak ji nabrus\dots\\
	\dots nabrus ji.
}

\vers{11}{
	A čím ji mám sbrousit\dots\\
	\dots řekni čím.
}

\vers{12}{
	Vem si brousek\dots\\
	\dots brousek.
}

\vers{13}{
	Jenže brousek je suchý\dots\\
	\dots suchý.
}

\vers{14}{
	Tak jej namoč\dots\\
	\dots smoč ho.
}

\vers{15}{
	A čím ho mám smáčet\dots\\
	\dots řekni čím.
}

\vers{16}{
	Zkus vodu\dots\\
	\dots vem si vodu.
}

\vers{17}{
	A v čem ji mám přinést\dots\\
	\dots řekni v čem.
}

\vers{18}{
	Vem si vědro\dots\\
	\dots vědro.
}

\vers{19}{
	Vědro má ve dně díru\dots\\
	\dots jak hrom.
}
\end{multicols}
\newpage
