\hyt{zumzumII}
\song{Zum zum II} \interpret{paveldobes}{Pavel Dobeš}
\ns

\vers{1}{
\chord{G}Zpívají o tom vrabci na Ro\chord{D\7}kytě, že učenec je horší nežli \chord{G}dítě.\\
\chord{G}Se žábami hraje si pan \chord{D\7}Galvani, Archimedes potápí se \chord{G}do vany\\
a, \chord{G}nepoučen událostmi \chord{D\7}ráje, Isaac Newton s jabkama si \chord{G}hraje.
}
\ns

\refrainn{1}{
\chord{G}Zum zum zum \chord{D\7}zum, a nejde mi to do kebule,\\
\chord{G}zum zum zum \chord{D\7}zum, a nejde mi to na ro\chord{G}zum.
}
\ns

\vers{2}{
Bylo to jak výbuj, jako salva, když se žárovkou přišel Thoma Alva.\\
Do pochodu vyhrávaly kapely, muži pili šampus, ženy šílely,\\
jak když pustíš tygry do arény, a začalo se dělat na tři směny.
} \refsm{1}
\ns

\vers{3}{
Kdyby naši předci vstali z ledu, podivili by se, jak jsme vpředu,\\
jak závazky předhánějí úkoly, Einstein by se těžko dostal na školy,\\
Mozart by moh' dneska u klavíru jen těžko dělat do muziky díru. \textbf{(G\7)}
}
\ns

\vers{4*}{\chord{C}Michelangelo by sebral \chord{G\7}sochy a hodil by je všecky do Ma\chord{C}cochy.\\
Lumiere by zčervenal jak \chord{G\7}malina, kdybyste ho vzali s sebou \chord{C}do kina.\\
Jen u elektrotechnického \chord{G\7}vesla, ještě ňákou dobu moh' by sedět \chord{C}Tesla. \chord{D\7}
}
\ns

\vers{5}{
Vědeckotechnická revoluce uvolňuje lidem obě ruce,\\
dnes má každý vědátor už od plínky sunarku a digitální hodinky.\\
S optimismem hledí k stratosféře a Brano samo zavírá mu dveře.
} \refsm{1}
\ns

\vers{6}{
Kdyby starý Thales nemoh' čmárat a kreslit si to phísku podle nálad,\\
Mendělejev kdyby musel, vážení, periodicky vykazovat hlášení\\
a osm hodin zvedat telefony, svět by stál za pytlík bikarbony.
} \refsm{1}
\ns

\vers{7}{
Nikdo z nás by doma neměl Sony dvakrát třicet wattů, čtyři ohmy.\\
Lidé by se hnali kamsi za hmotou, však regály by nejspíš zely prázdnotou.\\
neměli bychom šajn o opeře a válčilo by se u Sudoměře. \textbf{(G\7)}
}
\ns

\vers{8*}{Zem by byla rovná jako deska, nebyla by kulatá, jak dneska.\\
Adam s Evou nemuseli z ráje ven, Giordano Bruno by nebyl upálen,\\
jen temno, jak když vstoupíš do komory a škoda každé rány z Aurory.
}
\ns

\vers{9}{
Ze všech zvířat archy Noemovy a ze všeho, co můžem' popsat slovy,\\
jen balvany a lide mají odvahu urvat se od skály a padat dolů po svahu\\
a na světě, který se furt mění, překonat, co překonáno není.
}
\ns

\refrainn{2}{
Zum zum zum zum, protože to, co nejde do kebule,\\
zum zum zum zum, rádo leze na rozum.
}
\newpage
