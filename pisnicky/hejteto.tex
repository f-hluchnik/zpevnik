\hyt{hejteto}
\song{Hej teto!} \interpret{michalhorak}{Michal Horák}
\vers{1}{
Měl jsem \chord{Am}hlídat tetě \chord{C}křečka,\\
\chord{G}dala mi ho se slovy, ať \chord{D}u mě chvíli přečká.\\
Ale křečka uhlídati,\\
nelze jen tak lehce, obzvlášť, když se vám moc nechce.\\
Po pokoji ať si běhá,\\
na gauč jsem si lehnul, vzápětí se ani nehnul.\\
Tlačilo mě do zad kromě\\
polštářů i cosi, už se vidím tetu prosit:
}

\refrainn{1}{
Hej \chord{F}teto, já ho \chord{C}zabil,\\
\chord{E}už se to nestane, jo, \chord{Am}ten už asi nevstane.\\
Hej teto, nekřič na mě,\\
já ho prostě neviděl, už dost jsem se nestyděl!\\
Hej \chord{F}teto\dots\chord{C}\nc\chord{E}\nc\chord{Am}\nc\chord{F}\nc\chord{C}\nc\chord{E}\nc\chord{E\7}
}

\vers{2}{
Tak jsem kouknul, co mě tlačí,\\
seděl jsem jenom na televizním ovladači,\\
ale křeččí švitoření\\
slyšet není, jenom zvláštní smrad se line od topení.\\
Zase se mi dech zatajil,\\
spečeného křečka bych před tetou neobhájil.\\
A dřív než tam sebou seknu,\\
měl bych asi vymyslet, jak tetě potom řeknu:
}\refsm{1}

\refrainn{2}{
Hej \chord{F}teto, já ho fakt \chord{C}zabil,\\
ten \chord{E}bídák sežral magnet a k to\chord{Am}pení se přitavil.\\
Hej teto, co na to říci?\\
Můžeš si ho maximálně připnout na lednici.
}\refsm{1}

\newpage
