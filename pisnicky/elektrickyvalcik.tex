\hyt{elektrickyvalcik}
\song{Elektrický valčík}

\vers{1}{
\chord{Am}Jednoho letního večera na návsi pod starou \chord{E\7}lípou\\
\chord{E\7}hostinský Antonín Kučera vyvalil soudeček s \chord{Am}pípou.\\
Neby\chord{F}lo to posvícení, neby\chord{Am}la to neděle,\\
v naší \chord{F}obci mezi kopci plni\chord{E\7}ly se korbele.
}

\refrain{
\chord{A}Byl to ten slavný den, kdy k nám byl zaveden \chord{E\7}elekt\chord{E\dm}rický \chord{E\7}proud,\\
\chord{E\7}byl to ten slavný den, kdy k nám byl zaveden \chord{A}elekt\chord{A\dm}rický \chord{A}proud,\\
\rep{\chord{A\7}střída\chord{D}vý, \chord{E}střída\chord{C\kk m}vý,\sm\mm\chord{F\kk m}silný \sm \chord{Hm}elekt\chord{E\7}rický \chord{A}proud.}
}

\vspace{15pt}
\rec{A nyní, kdo tu všechno byl:\\
okresní a krajský inspektor, hasičský a recitační sbor,\\
poblíže obecní váhy tříčlenná delegace z Prahy,\\
zástupci nedaleké posádky pod vedením poručíka Vosátky,\\
početná skupina montérů (jeden z nich pomýšlel na dceru sedláka Krušiny),\\
dále krojované družiny, alegorické vozy, italský zmrzlinář Antonio Cosi,\\
na motocyklu Indián a svatý Jan, z kamene vytesán.} \refsm{}

\vers{2}{
Na stránkách obecní kroniky ozdobným písmem je psáno:\\
tento den pro zdejší rolníky znamená po noci ráno.\\
Budeme žít jako v Praze, všude samé vedení,\\
jedna fáze, druhá fáze, třetí pěkně vedle ní.
} \refsm{}

\vspace{15pt}
\rec{Z projevu inženýra Maliny, zástupce elektrických podniků:\\
Vážení občané, vzácní hosté, s elektřinou je to prosté.\\
Od pantáty vedou dráty do  žárovky nade vraty.\\
Odtud se proud přelévá do stodoly, do chléva.\\
Při krátkém spojení dvou drátů dochází k takzvanému zkratu.\\
Kdo má pojistky námi předepsané, tomu se při zkratu nic nestane.\\
Kdo si tam nastrká hřebíky, vyhoří a začne od píky.\\
Do každé rodiny elektrické hodiny!} \refsm{}
\newpage
