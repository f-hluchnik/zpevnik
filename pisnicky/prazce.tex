\hyt{prazce}
\song{Pražce} \interpret{paveldobes}{Pavel Dobeš}

\vers{1}{
\chord{A}Házím tornu na svý záda, feldtašku a \chord{E\7}sumky,\\
navštívím dnes kamaráda z železníční průmky.
}
\ns

\refrainn{1}{
Vždyť je \chord{A}jaro, zapni si kšandy, pozdravuj \chord{E\7}vlaštovky a, muziko, ty \chord{A}hraj.
}

\vers{2}{
Vystupuji z vlaku, který mizí v dálce,\\
stojím v České Třebové a všude kolem pražce.
} \refsm{1}

\vers{3}{
Pohostil mě slivovicí, představil mě Mařce,\\
posadil mě na lavici z dubového pražce.
} \refsm{1}

\vers{4}{
Provedl mě domem, nikdy kousek zdiva,\\
všude samej pražec, jen Máňa byla živá.
}
\ns

\refrainn{2}{
To je to jaro, zapni si kšandy, pozdravuj vlaštovky a muziko, ty hraj.
}

\vers{5}{
Plakáty nás informují, \uv{Přijď pracovat k dráze,\\
pakliže ti vyhovují rychlost, šmír a saze.}
} \refsm{1}

\vers{6}{
Jestliže jsi labužník a přes kapsu se praštíš,\\
upečeš i krávu na železničních pražcích.
} \refsm{1}

\vers{7}{
A naučíš se skákat, tak jak to umí vrabec,\\
když na nohu si pustíš železniční pražec.
} \refsm{1}

\vers{8}{
Když má děvče z Třebové rádo svého chlapce,\\
posílá mu na vojnu železniční pražce.
} \refsm{1}

\vers{9}{
A když děti zlobí, tak hned je doma mazec,\\
Děda Mráz jim nepřinese ani jeden pražec.
} \refsm{1}

\vers{10}{Před děvčaty z Třebové chlubil jsem se silou,\\
pozvedl jsem pražec, načež odvezli mě s kýlou.
} \refsm{1}

\vers{11}{Pamatuji pouze ještě operační sál,\\
pak praštili mě pražcem a já jsem tvrdě spal.
}
\ns

\refrainn{3}{
A bylo jaro, zapni si kšandy, lítaly vlaštovky a zelenal se háj.
}
\newpage
