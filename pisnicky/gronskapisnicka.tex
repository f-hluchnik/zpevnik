\hyt{gronskapisnicka}
\song{Grónská písnička} \interpret{nohavica}{Jaromír Nohavica}

\nv\textbf{D Em A\7 D}

\vers{1}{
Daleko na severu je Grónská zem,\\
žije tam Eskymačka s Eskymákem.\\
\rep{My bychom umrzli, jim není zima,\\
snídají nanuky a eskyma.}
}

\vers{2}{
Mají se bezvadně, vyspí se moc,\\
půl roku trvá tam polární noc.\\
\rep{Na jaře vzbudí se a vyběhnou ven,\\
půl roku trvá tam polární den.}
}

\vers{3}{
Když sněhu napadne nad kotníky,\\
hrávají s medvědy na četníky.\\
\rep{Medvědi těžko jsou k poražení,\\
neboť medvědy ve sněhu vidět není.}
}

\vers{4}{
Pokaždé ve středu, přesně ve dvě,\\
zaklepe na íglů hlavní medvěd:\\
\rep{\uv{Dobrý den, mohu dál na vteřinu?\\
Nesu vám trochu ryb na svačinu.}}
}

\vers{5}{
V kotlíku bublá čaj, kamna hřejí,\\
psi venku hlídají před zloději.\\
\rep{Smíchem se otřásá celé íglů,\\
medvěd jim předvádí spoustu fíglů.}
}

\vers{6}{
Tak žijou vesele na severu,\\
srandu si dělají z teploměrů.\\
\rep{My bychom umrzli, jim není zima,\\
neboť jsou doma a mezi svýma.}
}
\newpage
