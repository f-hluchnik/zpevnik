\hyt{letniromance}
\song{Letní romance}

\vers{1}{
Je \chord{G}ráno, \chord{C}pěknej \chord{G}den jako \chord{C}neje\chord{G}den a \chord{C}brána k štěstí \chord{G}otev\chord{D}řená.
\\
A v \chord{G}světle \chord{C}přikry\chord{G}tá vedle \chord{C}leží \chord{G}ta, která \chord{C}je a není \chord{G}moje \chord{D}žena.\\
Ať \chord{C}žije lidskej \chord{G}tvor, je \chord{C}osmej thermi\chord{G}dor\footnote{Jedenáctý měsíc francouzského revolučního kalendáře z 18. století. Zároveň tak bývá nazýváno spiknutí, které 27. července (9. thermidoru) 1794 svrhlo revoluční vládu Maximiliena Robespierra a nastolilo vládu velké finanční a průmyslové buržoazie. Osmého thermidoru pronesl Robespierre projev plný výzev a hrozeb.} a \chord{C}lid za oknem \chord{G}volá \chord{D}sláva\\
\rep{A \chord{G}ona \chord{C}po chví\chord{G}li v mojí \chord{C}koši\chord{G}li \chord{C}vstává, \chord{G}tiše \chord{D}vstá\chord{G}vá.}
}

\vers{2}{
Pod blůzou z Marseille však má dvě kameje hezčí, než má senešálka\footnote{Senešal (z latinského senescalcus – nejstarší sluha) byl titul vysokého úředníka ve středověké Francii. V raném středověku to byl zkušený a spolehlivý člen panovnické družiny, který působil jako stolník. Postupně získal pravomoc nad všemi palácovými záležitostmi a uplatňoval také politický vliv.}.\\
A já je propíjím s jednou bestií, která je a není moje válka.\\
Jó, v malým pokoji ve tmě a ve zbroji se rázem kroků nedostává.\\
\rep{Když země kdekoli je v letní řeholi plavá, tolik plavá.}
}

\vers{3}{
Léto uzraje a tahle holka je moc bílá na to, aby žila\\
jen ze svých kamejí a s mojí nadějí, v který je a není moje síla.\\
\rep{Jó, je mi do tance, tahle romance je možná ženská nebezpečná.\\
Možná krvavá, možná nepravá, ale věčná, Bože, věčná.}
}
