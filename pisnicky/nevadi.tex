\hyt{nevadi}
\song{Nevadí} \interpret{wabidanek}{Wabi Daněk}

\vers{1}{
\chord{A}Napůl jako hrou a \chord{E}napolovic vážně, čerstvou \chord{D}maturitou zmužn\chord{A}ělý\chord{E\bas{(E, F\kk A H C\kk E C\kk H)}},\\
\chord{A}chtěl jsi dobýt svět \chord{E}a svět se tvářil vlažně na tvý \chord{D}bubnování nesmě\chord{A}lý. oo\chord{E}o\\
\chord{A}Hlavu plnou věd a \chord{E}nadějí svý mámy, zkoušky \chord{G}přijímací, \chord{D}potom \chord{A}pá\chord{E\bas{(E, F\kk A H C\kk E C\kk H)}}d.\\
\chord{A}Znalosti jsi měl, cos \chord{E}neměl, byli známí, takže \chord{D}litujeme, za rok \chord{A}snad.\\
\chord{Hm}A tehdy \chord{Hm\maj}poprvé jsi řekl \chord{Hm\7}\sm neva\chord{E}dí, zase \chord{A}bude líp.
}

\vers{2}{
Pomalu šel rok a zase stejná škola, čekáš předvolání každej den.\\
Předvolání máš a na něm vlast tě volá, takže za dva roky, čert to vem.\\
Dva roky vzal čert a za nějakej týden stojíš na radnici s holkou svou.\\
Bylo to s ní fajn, tys věřil, že to vyjde. Kde tvý nebetyčný plány jsou?\\
A tehdy podruhé jsi řekl, nevadí, zase bude líp.
}

\vers{3}{
Neměli jste byt a vlastně ani prachy, prej, že ve dvou se to táhne líp.\\
Jenže už jste tři a máma s dvěma bráchy, starej dvoupokoják, špatnej vtip.\\
Tak jste žili rok v tý supertěsný kleci, vzteky vysušený jako troud.\\
Potom, jeden den, si žena vzala věci, další připomínky řešil soud.\\
A tehdy potřetí jsi řekl, nevadí, zase bude líp.
}

\vers{4}{
Už jsi dlouho sám a věci, co se stanou, se tě nedotýkaj', byl jsi bit.\\
Všechno už jsi vzdal a postavil se stranou, vzal sis dovolenou, chceš mít klid.\\
Osamělý dům u rychlíkové trati, oči za záclonou zapranou.\\
Vlaky jedou dál a málokdy se vrátí, život nemá brzdu záchrannou.\\
Tak řekni, sakra, to svý nevadí, zase bude líp!
}

\newpage
