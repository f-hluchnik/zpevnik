\hyt{chaloupka}
\song{Chaloupka}
\vers{1}{
Vysta\chord{G}vím si \chord{C}skromnou \chord{G}chaloupku,\\
všecko \chord{D\7}čistě, jako \chord{G}na sloupku,\\
stromo\chord{G}ví a \chord{C}křoví \chord{G}vůkol ní, budou \chord{D\7}dávat chládek k ově\chord{G}ní.\\
Odtud \chord{D\7}nedaleko \chord{G}k háji, besíd\chord{D\7}ku si spletu \chord{G}s májí,\\
až to \chord{G}všechno \chord{C}řádně \chord{G}dovedu, pak tě \chord{D\7}dívčinko tam pove\chord{G}du.
}

\vers{2}{
V chlívku bude s kozou kravička a u sroubku z drnu lavička,\\
stáj a ovčín v jednom pořadí, pod kolnou pak orní nářadí.\\
Stodůlka a šest kroků sáze, vůkol plot a kousek hráze,\\
až to všechno řádně dovedu, pak tě dívčinko tam povedu.
}

\vers{3}{
Na zahrádce mimo zeliny, budou také vonné květiny,\\
podzim rozličného ovoce, bude víc a víc rok po roce.\\
Ptactvo bude houští cvrčet a potůček veskrz hrčet,\\
až to všechno řádně dovedu, pak tě dívčinko tam povedu.
}

\vers{4}{
na dvorečku kurník kulatý, pro svou radost kvočnu s kuřaty,\\
kachnám, husám a vší drůbeži, sama budeš sypat večeři.\\
Něco pro dům, něco v Praze, na trhu se prodává-vavalo draze,\\
až to všechno řádně dovedu, pak tě dívčinko tam povedu.
}

\vers{5}{
Jizbu prostou, sklípek klenutý, na mléko a soudek přepnutý,\\
z něhož, když k nám známí doskočí, džbánek čerstvého se natočí.\\
A v komůrce bude lůže, ve kterém se dvé směstnat může,\\
až to všechno řádně dovedu, pak tě dívčinko tam povedu.
}

\vers{6}{
Jestli se ti to tak zalíbí, pak tě hoch tvůj vroucně políbí,\\
pak se zeptá, budeš-li jej chtít, byť i o vše přišel, ráda mít.\\
A až potom budem' svoji, pak nás žádný nerozdvojí,\\
až to všechno řádně dovedu, pak tě dívčinko tam odvedu.
}
\newpage
