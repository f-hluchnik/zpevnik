\hyt{dobrodruzstvisbohempanem}
\song{Dobrodružství s bohem Panem} \interpret{kubisova}{Marta Kubišová}
\hyl{greensleeves}{Greensleeves}
\hyl{romance16leta}{Romance 16. léta}

\vspace{15pt}
\note{capo 3}

\vers{1}{
Je \chord{Am}půlnoc \chord{C}nádherná, \chord{G}spí i \chord{Em}lucerna, \chord{Am}tys' mě opustil \chord{E}ospalou.\\
Tu \chord{Am}v hloubi \chord{C}zahrady \chord{G}cítím \chord{Em}úklady, \chord{Am}s píšťalou \chord{E\7}někdo sem \chord{Am}kráčí.\\
\chord{C}Hrá náramně \chord{G}krásně \chord{Em}a na mě \chord{Am}tíha podivná \chord{E}doléhá.\\
\chord{C}Hrá náramně, \chord{G}zná mě, \chord{Em}nezná mě, \chord{Am}něha a \chord{E\7}hudba až \chord{Am}k pláči.
}

\vers{2}{
Pak náhle pomalu skládá píšťalu, krok, a slušně se uklání.\\
Jsem rázem ztracená, co to znamená? Odháním strach a on praví:\\
Pan, jméno mé, mám už renomé, Pan se jmenuju a jsem bůh.\\
Pan, bůh všech stád, vás má, slečno, rád, jen Pan je pro vás ten pravý.
}

\vers{3}{
Ráno, raníčko, ach, má písničko, Pan mi zmizel i s píšťalou.\\
Od Pana, propána, o vše obrána, ospalou najde mě máti.\\
Hrál a ve tmě krásně podved' mě, kam jsem to dala oči, kam?\\
Pan, pěkný bůh, já teď nazdařbůh počítám \uv{dal} a \uv{má dáti}.
}
\newpage
