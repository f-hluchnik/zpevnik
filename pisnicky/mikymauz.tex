\hyt{mikymauz}
\song{Mikymauz} \interpret{nohavica}{Jaromír Nohavica}\interpretdva{plihal}{Karel Plíhal}

\large
\vspace{-8pt}

\vers{1}{
\chord{Am}Ráno mě probouzí \chord{G}tma, sahám si \chord{Em\7}na zápěstí, \chord{F}zda mi to \chord{E}ještě tluče, \chord{F}zdali mám \chord{E}ještě štěstí,\\
\chord{Am}nebo je po mně a \chord{G}já mám vosko\chord{Em\7}vané boty, \chord{F}ráno co \chord{E}ráno stejné \chord{F}probuzení \chord{E}do nico\chord{Am}ty.
}
\vspace{-8pt}

\vers{2}{
Není co, není proč, není jak, není kam, není s kým, není o čem, každý je v sobě sám.\\
Vyzáblý Don Quijote sedlá svou Rosinantu a Bůh je slepý řidič sedící u volantu.
}
\vspace{-8pt}

\refrainn{1}{
\chord{E}Zapínám \chord{F}telefon \chord{Dm}záznamník \chord{E}cizích citů, \chord{E}špatné zprávy \chord{F}chodí jako \chord{Dm}policie \chord{E}za úsvitu.\\
\chord{Dm}Jsem napůl bdělý a \chord{G}napůl ještě v noční pauze, \chord{C}měl bych se smát, ale \chord{E\7}mám úsměv Mikymauze,\\
\chord{Am}rána \chord{G\add{9}}bych zru\chord{F\maj}šil. \chord{Em\7}\nc\chord{Am}
}
\vspace{-8pt}

\vers{3}{
Nějaký dobrák v rádiu pouští Chick Coreu\footnote{Armando Anthony \uv{Chick} Corea byl americký jazzový pianista a skladatel. Hrál společně s Milesem Davisem na desce Bitches Brew, v roce 1968 nahrál s českým kontrabasistou Miroslavem Vitoušem album Now He Sings, Now He Sobs.}, opravdu veselo je, asi jako v mauzoleu.\\
Ve frontě na mumii mám kruhy pod očima, růžový rozbřesk fakt už mě nedojímá.
}
\vspace{-8pt}

\vers{4}{
Povídáš něco o tom, co bychom dělat měli. Pomalu vychládají naše důlky na posteli.\\
Všechno se halí v šeru, čí to bylo vinou, že dřevorubec máchl mezi nás širočinou?
}
\vspace{-8pt}

\refrainn{2}{
Postele rozdělené na dva suverénní státy, ozdoby na tapetách jsou jak pohraniční dráty.\\
Ve spánku nepřijde to, spánek je sladká mdloba, že byla ve mně láska, je jenom pustá zloba,\\
dráty bych zrušil.
}
\vspace{-8pt}

\vers{5}{
Prokletá hodina, ta minuta, ta krátká chvíle, kdy věci nejsou černé, ale nejsou ani bílé,\\
kdy není tma, ale ještě ani vidno není. Bdění je bolest bez slastného umrtvení.
}
\vspace{-8pt}

\vers{6}{
Zběsile mi to tepe a tupě píchá v třísle, usnout a nevzbudit se, nemuset na nic myslet.\\
Opřený o kolena poslouchám tvoje slzy, na život už je pozdě a na smrt ještě brzy.
}
\vspace{-8pt}

\refrainn{3}{
Co bylo kdysi, včera, je jako by nebylo by, káva je vypita a není žádná do zásoby,\\
věci, co nechceš, ať se stanou, ty se stejně stanou a chleba s máslem padá na zem vždycky blbou stranou,\\
máslo bych zrušil.
}
\vspace{-8pt}

\vers{7}{
Povídáš o naději a slova se ti pletou, jak špionážní družice letící nad planetou.\\
Svlíknout se z pyžama, to by šlo ještě lehce. Dvacet let mluvil jsem a teď už se mi mluvit nechce.
}
\vspace{-8pt}

\vers{8}{
Z plakátu na záchodě prasátko vypasené kyne mi, zatímco se kolem voda dolů žene.\\
Všechno je vyřčeno a odnášeno do septiku, jenom mně tady zbývá prodýchat pár okamžiků.
}
\vspace{-8pt}

\refrainn{4}{
Sahám si na zápěstí a venku už je zítra, hodiny odbíjejí signály dobrého jitra.\\
Jsem napůl bdělý a napůl ještě v noční pauze, měl bych se smát, ale mám úsměv Mikymauze,\\
lásku bych zrušil.
}
\vspace{-8pt}

\cod{
\rep{Ráno mě probouzí tma, sahám si na zápěstí, zda mi to ještě tluče, zdali mám ještě štěstí.}
}
\Large
\newpage
