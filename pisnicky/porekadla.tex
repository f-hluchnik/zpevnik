\hyt{porekadla}
\song{Pořekadla} \interpret{wabidanek}{Wabi Daněk}
\ns

\vers{1}{
\chord{G}Rád bych se zeptal \chord{H\7}těch, kteří vědí, \chord{Em}proč místo zlatem \chord{C}platíme mědí?\\
\chord{G}Proč nejsme bílí, a \chord{D}proč jsme jen šedí? \chord{C}Ti, co chytrou \chord{D}kaši je\chord{G}dí,\mm\chord{Em}\\
\chord{C}určitě mi \chord{D}odpově\chord{G}dí, \chord{C}\nc\chord{G}
}

\refrainn{1}{
Že \chord{C}z nouze si žijem a \chord{Hm}milujem z nouze, \chord{H\7}zpíváme málo a \chord{Em}kecáme \chord{C}dlouze\\
a \chord{G}tváře si myjem v \chord{D}blátivý strouze, \chord{C}když se čistá \chord{D}neseže\chord{G}ne, \chord{Em}\\
\chord{C}ani v lese \chord{D}u prame\chord{G}ne. \chord{C}\nc\chord{G}
}

\vers{2}{
Tak dlouho se džbánem, až ucho upadne, jablko od stromu daleko nepadne,\\
než holub na střeše, líp vrabec v hrsti -- tohle mi jde proti srsti,\\
zatraceně proti srsti.
}

\refrainn{2}{
Z nouze si žijem a milujem z nouze, zpíváme málo a kecáme dlouze\\
a tváře si myjem v blátivý strouze, když se čistá nesežene,\\
nesežene, a ne že ne.
}

\vers{3}{
Řekni mi, holka, čí je to vinou, že ty chceš jiného, on zase jinou,\\
čím je to daný, že ti praví se minou, ti nepraví, že se berou,\\
potom se div nesežerou.
}

\refrainn{3}{
\chord{C}Z nouze si žijem a \chord{Hm}milujem z nouze, \chord{H\7}zpíváme málo a \chord{Em}kecáme \chord{C}dlouze\\
a \chord{G}tváře si myjem v \chord{D}blátivý strouze, \chord{C}když se čistá \chord{D}neseže\chord{G}ne,\sm\chord{Em}\\
\chord{C}je to pravda, \chord{D}a ne že \chord{G}ne.\chord{Em}
}
\ns

\cod{
Že \chord{C}cesty už jsou \chord{D}vychoze\chord{G}né,\sm\chord{Em}\nc a \chord{C}uzené je \chord{D}vyuze\chord{G}né,\sm\chord{Em}\\
a \chord{C}pivo dobře \chord{D}vychlaze\chord{G}né,\sm\chord{Em}\nc \chord{C}žádná nouze \chord{D}vlastně ne\chord{G}ní,\sm\chord{Em}\\
tak \chord{C}načpak tohle \chord{D}pozdviže\chord{G}ní. \chord{C}\nc\chord{G}
}
\newpage
