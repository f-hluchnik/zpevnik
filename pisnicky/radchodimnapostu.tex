\hyt{radchodimnapostu}
\song{Rád chodím na poštu} \interpret{pokac}{Pokáč}
\note{originál v F dur (D dur, capo 3)}

\refrainn{1}{
\chord{G}Rád chodím \chord{D}na poštu \chord{C}nejsem tam jen \chord{G}do počtu,\\
\chord{C}jsem tam důle\chord{G}žitý člen \chord{Am}fronty, která \chord{D}trčí ven.\\
\chord{G}Vždy, když se \chord{D}cítím sám, \chord{C}rád na poštu \chord{G}zavítám,\\
\chord{C}tam je tolik \chord{G}lidí, že \chord{Am}pohnout se mám \chord{D}potí\chord{G}že.
}

\vers{1}{
\chord{C}Za přepážkou \chord{G}paní v letech \chord{C}má razítko \chord{G}gumové\\
a \chord{C}občas tiskne \chord{Em}tiskárnou z dob \chord{Am}první války \chord{D}světové.\\
Každému, kdo spílá jí, že zas na poště prosral den,\\
nabídne los stírací, ať uklidní se hazardem.
}\refsm{1}

\vers{2}{
Proč je vždy jen jedna z pěti přepážek otevřená\\
je záhada, co navždy zůstat má tajemstvím zastřená.\\
Má nejoblíbenější služba je balíček do ruky,\\
znamená to totiž, že zas budu moct jít na poštu,\\
jestli ho vůbec doručej', to už je bez záruky.
}

\refrainn{2}{
Já rád chodím na poštu, nejsem tam jen do počtu,\\
jsem tam důležitý člen fronty, která trčí ven.\\
Vždy, když tam stepuju, tenhle song si notuju,\\
mám pak hnedka trochu míň chuť do mozku si vrazit klín.
}
\newpage
